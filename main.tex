\documentclass[conference,10pt]{IEEEtran}
\IEEEoverridecommandlockouts
\usepackage[utf8]{inputenc}
\usepackage{lastpage}
\usepackage{mathptmx}
\usepackage[dvipsnames]{xcolor}
\usepackage[T1]{fontenc}
\usepackage[english]{babel}
\usepackage{array}
\usepackage{url}
\usepackage{graphicx}
\usepackage{flushend}
\usepackage{cite}
\usepackage{algorithmic}
\usepackage[caption=false]{subfig}
\usepackage{hyperref}
\usepackage{microtype}

\def\UrlBreaks{\do\/\do-}

\newcommand{\papertitle}{SenSQL}

% Potential venues:
%
% IFIP Networking 2021 (~ 23\%)
%  * Deadlines:
%    - Abstract: January 5, 2021 (https://edas.info/N27861)
%    - Paper: January 12, 2012
%  * Guidelines:
%    - 9 pages maximum
%    - double-blind review
%
% IEEE DCOSS 2021 (~ 25-30\%)
%  * Deadlines:
%    - Abstract: January 18, 2021 ( https://edas.info/newPaper.php?c=27882)
%    - Paper: January 25, 2021
%  * Guidelines:
%    - 8 pages maximum (2 additional at $100/page for appendices, theorems, implementation)
%    - single-blind review

\hypersetup{
  colorlinks=true,% Color URLs with urlcolor
  linkcolor=black,% The color for internal (cross-reference) links
  urlcolor=blue,% The color for URLs (hyperlinks)
  citecolor=black,
  pdftitle={\papertitle}
}

\urlstyle{same}

\begin{document}
\title{\papertitle}

\author{
  \IEEEauthorblockN{
    \href{mailto:janakj@cs.columbia.edu}{\color{black}Jan Janak},
    \href{mailto:hgs@cs.columbia.edu}{\color{black}Henning Schulzrinne}
  }
  \IEEEauthorblockA{
    Department of Computer Science, Columbia University, New York, USA
  }
  Email: \{janakj,hgs\}@cs.columbia.edu
}

\maketitle

\begin{abstract}
\end{abstract}

\section{Introduction}
\label{sec:introduction}

Many Internet of Things (IoT) devices include sensors that produce measurements. Where are the measurements stored and how does an application query such data? In this paper, we propose a SQL-based distributed storage and query processing system for IoT applications.

\section{Related Work}
\label{sec:related-work}

The use of SQL was explored in the context of Wireless Sensor Networks (WSN).

\nocite{madden2005tinydb}
\nocite{madden2002supporting}
\nocite{bacon2017spanner}
\nocite{sun2010querying}
\nocite{rfc5222}
\nocite{rfc5582}
\nocite{rfc5012}
\nocite{rfc8428}

\bibliographystyle{IEEEtran}
\bibliography{bibs/references,bibs/rfc}
\end{document}
